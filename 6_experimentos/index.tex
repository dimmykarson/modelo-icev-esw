\section{Experimentos}\label{:experiments}

Na seção de experimentos de um artigo, você descreve os detalhes específicos dos experimentos ou estudos realizados como parte da sua pesquisa. Essa seção fornece informações sobre como você coletou os dados e realizou as análises. Aqui estão algumas informações que você pode incluir na seção de experimentos:

\subsection{Objetivos do experimento}
Declare claramente os objetivos do experimento ou estudo em questão. Explique o que você pretende investigar, medir ou comparar por meio do experimento.

Variáveis independentes e dependentes: Identifique as variáveis independentes (aquelas que você manipula ou controla) e as variáveis dependentes (aquelas que você mede ou observa em resposta à manipulação). Descreva as unidades de medida e como as variáveis foram operacionalizadas.

Descreva em detalhes os procedimentos específicos que foram seguidos durante o experimento. Explique como você implementou a manipulação das variáveis independentes, a coleta de dados e qualquer intervenção realizada. Inclua informações sobre o design experimental, a ordem das condições, as tarefas ou estímulos apresentados aos participantes e qualquer etapa de pré-processamento de dados.

\subsection{Instrumentos e materiais}
Descreva os instrumentos, questionários, escalas de medição, equipamentos ou materiais específicos utilizados no experimento. Explique sua validade e confiabilidade, se aplicável, e como eles foram administrados ou apresentados aos participantes.

\subsection{Procedimentos de análise de dados}
Descreva os procedimentos de análise de dados que foram aplicados aos resultados do experimento. Explique as técnicas estatísticas utilizadas para analisar os dados e como elas foram escolhidas para responder às suas perguntas de pesquisa. Mencione quaisquer pressupostos estatísticos e como você tratou dados ausentes ou atípicos.

Certifique-se de que a descrição dos experimentos seja clara e suficientemente detalhada para que outros pesquisadores possam entender e replicar seu estudo. Inclua referências adequadas para instrumentos, escalas ou materiais utilizados, se necessário.