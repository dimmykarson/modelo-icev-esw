\documentclass[conference]{IEEEtran}
\IEEEoverridecommandlockouts

%Imports necessários!
\usepackage{cite}
\usepackage{amsmath,amssymb,amsfonts}
\usepackage{algorithmic}
\usepackage{graphicx}
\usepackage{textcomp}
\usepackage{xcolor}
\usepackage[brazil]{babel}
\usepackage{natbib}
\usepackage{fancyhdr}


\def\BibTeX{{\rm B\kern-.05em{\sc i\kern-.025em b}\kern-.08em
    T\kern-.1667em\lower.7ex\hbox{E}\kern-.125emX}}
\newcommand{\icev}{Faculdade iCEV}
%VARIÁVEIS RELACIONADAS AO TRABALHO
\newcommand{\titulo}{Título do Trabalho}
\newcommand{\palavraschavespt}{componente, formatação, estilo}
\newcommand{\palavraschavesen}{component, formatting, style, styling, inser}


%Dados do aluno
\newcommand{\aluno}{Fulano de Tal da Silva}
\newcommand{\emailAluno}{fulado.silva@somosicev.com}

%Dados do orientador
\newcommand{\orientador}{Ciclano de Tal da Silva}
\newcommand{\emailOrientador}{ciclano.silva@somosicev.com}

%Dados do curso
\newcommand{\escola}{Escola de Tecnologia}
\newcommand{\curso}{Engenharia de Software}
\newcommand{\cidade}{Teresina}
\newcommand{\uf}{PI}

\usepackage{fancyhdr}

\begin{document}

\pagestyle{fancy}
\fancyhf{}
\rhead{\titulo}
\lhead{\bfseries \icev}
\rfoot{}

\title{
    \includegraphics[width=.2\textwidth]{images/logoicev.png}
    \\\titulo}

\author{\IEEEauthorblockN{\aluno}
\IEEEauthorblockA{\textit{\escola} \\
\textit{\curso}\\
\cidade, \uf \\
\emailAluno}
\and
\IEEEauthorblockN{\orientador}
\IEEEauthorblockA{\textit{\escola} \\
\textit{\curso}\\
\cidade, \uf \\
\emailOrientador
}}

\maketitle


\textit{\textbf{Resumo}} - 
\textbf{Este é o resumo do seu trabalho em português. Inclua aqui uma breve síntese do seu estudo, destacando os objetivos, métodos utilizados, resultados e conclusões.}

\textit{\textbf{Palavras-chaves - 
\palavraschavespt}}

\section{Introdução}

A introdução de um artigo é uma seção importante que estabelece o contexto, a relevância e os objetivos do estudo. Ela deve fornecer ao leitor uma visão geral do tema, despertando seu interesse e estabelecendo as bases para o restante do artigo. Aqui estão alguns elementos que geralmente são incluídos em uma introdução:

\subsection{Contextualização}
Apresente o tema geral do artigo, fornecendo informações de fundo e destacando a importância do assunto. Explique por que o tema é relevante e qual a relevância acadêmica ou prática do estudo.

\subsection{Revisão da literatura}
Faça uma breve revisão dos trabalhos anteriores e das pesquisas relacionadas ao tema do seu estudo. Isso ajudará a situar seu trabalho dentro do contexto existente, mostrando como sua pesquisa se diferencia ou se baseia em estudos anteriores.

\subsection{Lacuna na literatura}
Identifique uma lacuna na pesquisa existente que seu estudo pretende preencher. Descreva como seu trabalho se encaixa nessa lacuna e como ele pode contribuir para o conhecimento atual sobre o assunto.

\subsection{Objetivos da pesquisa}
Declare claramente os objetivos e as perguntas de pesquisa do seu estudo. Explique o que você pretende alcançar e quais questões você está tentando responder. Essa parte deve ser específica e concisa.

\subsection{Estrutura do artigo}
Forneça uma breve visão geral da estrutura do artigo, descrevendo como as seções subsequentes estão organizadas e o que será abordado em cada uma delas. Isso ajudará o leitor a entender a organização do seu trabalho. Na Seção \ref{:relatedwork} possui uma revisão mais detalhada dos estudos anteriores e das pesquisas relacionadas ao tema do seu estudo. A Seção \ref{:methodoloy} descreve o processo que foi seguido para conduzir o estudo ou pesquisa. A Seção \ref{:approach} descreve a perspectiva teórica, a estrutura conceitual ou a abordagem metodológica que você adotou para abordar seu objeto de estudo. Em seguida a seção \ref{:experiments} descreve-se os detalhes específicos dos experimentos ou estudos realizados como parte da pesquisa. Na Seção \ref{:results} apresenta-se e descreve-se os principais achados ou descobertas da pesquisa. Uma interpretação dos resultados do estudo, relaciona-os ao contexto mais amplo da pesquisa e suas implicações teóricas e práticas são discutidos na Seção \ref{:discussion}. Por fim, na Seção \ref{:conclusion}  resume-se os principais pontos discutidos ao longo do estudo e destaca as contribuições e implicações mais importantes da pesquisa.

É importante lembrar que a introdução deve ser clara, sucinta e direta ao ponto. Evite informações desnecessárias ou excessivamente detalhadas. Seja objetivo ao estabelecer o contexto e garantir que a introdução seja interessante o suficiente para cativar a atenção do leitor.

\section{Trabalhos Relacionados}\label{:relatedwork}

A seção Trabalhos Relacionados ("Related Work") em um artigo tem como objetivo fornecer uma revisão mais detalhada dos estudos anteriores e das pesquisas relacionadas ao tema do seu estudo. Essa seção demonstra que você está familiarizado com a literatura existente e ajuda a posicionar seu trabalho em relação a outros estudos.

\subsection{Visão geral da literatura} 
Inicie com uma visão geral da literatura relevante. Identifique os principais tópicos, abordagens ou teorias discutidas na literatura e explique como essas pesquisas são relevantes para o seu estudo.

\subsection{Trabalhos relacionados}
Discuta estudos anteriores que são diretamente relevantes para o seu trabalho. Descreva brevemente esses estudos, destacando seus objetivos, métodos e principais descobertas. Compare e contraste esses estudos com o seu, ressaltando como o seu trabalho se diferencia ou complementa o que já foi feito. Todos os artigos devem estar referenciados como aqui \cite{b2}. As citações são pontos importantes do trabalho, por que demonstra que o aluno se aprofundou no tema estudado \cite{b4}.

\subsection{Lacunas e limitações}
Identifique as lacunas na literatura existente ou as limitações dos estudos anteriores. Mostre como seu trabalho pretende preencher essas lacunas ou superar essas limitações. Isso ajuda a justificar a importância do seu estudo e a destacar sua contribuição para o campo.

\paragraph{Dica importante}: Dê destaque a estudos recentes que foram publicados na área do seu tema. Demonstre como esses estudos estão alinhados com o seu trabalho e como seu estudo se baseia ou expande essas pesquisas mais recentes.

É importante ser completo ao revisar a literatura relacionada, citando adequadamente os estudos relevantes e fornecendo referências bibliográficas. Isso demonstra que você está construindo seu trabalho sobre uma base sólida de conhecimento existente e contribuindo para a conversa acadêmica.

\section{Metodologia}\label{:methodoloy}

A seção de metodologia em um artigo descreve o processo que você seguiu para conduzir o seu estudo ou pesquisa. Ela deve ser detalhada o suficiente para permitir que outros pesquisadores reproduzam o seu trabalho e avaliem a validade dos resultados. Aqui estão algumas informações que você pode incluir na seção de metodologia:

\subsection{Desenho do estudo}
Descreva o desenho do estudo, ou seja, a abordagem geral adotada para coletar dados e responder às perguntas de pesquisa. Por exemplo, você pode mencionar se o estudo é qualitativo, quantitativo, experimental, observacional, etc.

Participantes ou amostra: Descreva as características dos participantes ou da amostra envolvida no estudo. Inclua informações como o tamanho da amostra, critérios de inclusão e exclusão, método de seleção dos participantes, e quaisquer outras informações relevantes sobre as pessoas envolvidas no estudo.

Coleta de dados: Explique como os dados foram coletados. Isso pode incluir detalhes sobre os instrumentos de coleta de dados utilizados, como questionários, entrevistas, observações ou medidas objetivas. Descreva também o local e o período de coleta de dados.

\subsection{Procedimentos}
Descreva os procedimentos específicos seguidos durante o estudo. Isso pode incluir instruções dadas aos participantes, detalhes sobre as condições experimentais, sequência de atividades, cronograma e qualquer intervenção realizada.

Variáveis e medidas: Liste e descreva as variáveis do estudo, especificando como elas foram medidas ou operacionalizadas. Explique as escalas de medição utilizadas, bem como a validade e confiabilidade dos instrumentos de medição, quando aplicável.

\subsection{Análise de dados}

Descreva as técnicas estatísticas ou métodos de análise de dados que foram aplicados aos dados coletados. Explique como as variáveis foram analisadas e quaisquer suposições subjacentes aos métodos estatísticos utilizados.


\subsection{Informações sobre o artigo}

\subsubsection{Autores e Afiliações}
\textbf{O arquivo de classe é projetado para, mas não limitado a, seis autores.} É necessário no mínimo um autor. Os nomes dos autores devem ser listados da esquerda para a direita e, em seguida, descendo para a próxima linha. Essa sequência de autores será usada em futuras citações e pelos serviços de indexação. Os nomes não devem ser listados em colunas ou agrupados por afiliação.

\section{Abordagem}\label{:approach}

Na seção de abordagem de um artigo, você descreve a perspectiva teórica, a estrutura conceitual ou a abordagem metodológica que você adotou para abordar seu objeto de estudo. Essa seção permite que os leitores compreendam o enfoque teórico ou prático subjacente ao seu trabalho. A seguir estão algumas informações que você pode incluir na seção de abordagem.

\subsection{Perspectiva teórica}
Descreva a perspectiva teórica que fundamenta seu estudo. Explique o arcabouço conceitual ou a teoria que serve de base para a pesquisa. Destaque os principais conceitos, modelos ou teorias que você está utilizando para analisar ou interpretar seus dados.

Se o seu estudo envolve a construção de uma estrutura conceitual ou modelo, explique-o nesta seção. Descreva os elementos-chave da estrutura e explique como eles se relacionam entre si. Mostre como essa estrutura conceitual ajudará a responder às perguntas de pesquisa ou a abordar o problema proposto.

\subsection{Metodologia adotada} Se o seu estudo é orientado para a prática ou possui uma abordagem metodológica específica, explique-a nesta seção. Por exemplo, se você está usando uma abordagem de estudo de caso, pesquisa-ação, pesquisa participativa ou outra metodologia específica, descreva como essa abordagem orientou sua coleta de dados e análise.

Explique por que você escolheu essa abordagem em particular e como ela se alinha com seus objetivos de pesquisa. Discuta as vantagens e as limitações dessa abordagem específica em relação ao seu estudo e como ela pode contribuir para responder às suas perguntas de pesquisa.

É importante fornecer detalhes suficientes para que os leitores possam compreender claramente a abordagem teórica ou metodológica que você adotou. Use exemplos ou ilustrações, quando apropriado, para ajudar a esclarecer a abordagem e sua relevância para o estudo.

\subsection{Figuras e Tabelas}
\paragraph{Posicionamento de Figuras e Tabelas} Posicione as figuras e tabelas no topo e na parte inferior das colunas. Evite colocá-las no meio das colunas. Figuras e tabelas grandes podem se estender por ambas as colunas. As legendas das figuras devem estar abaixo das figuras; os cabeçalhos das tabelas devem aparecer acima das tabelas. Insira figuras e tabelas após serem mencionadas no texto. Use a abreviação "Fig.~\ref{fig:figbase}" mesmo no início de uma frase.

É importante ilustrar sua abordagem com imagens sempre que apropriado. Imagens, como gráficos, tabelas, diagramas ou esquemas, podem ser uma forma eficaz de visualizar conceitos complexos, padrões de dados, relações entre variáveis ou processos metodológicos. As imagens devem ser sempre referencias como quando usamos a Figura \ref{fig:figbase}.

\begin{figure}[htbp]
    \centering
    \includegraphics{images/fig1.png}
    \caption{Nunca esqueça de colocar uma legenda para cada imagem associada ao seu trabalho.}
    \label{fig:figbase}
\end{figure}

\subsection{Equações}
Numere as equações consecutivamente. Para tornar suas equações mais compactas, você pode usar a barra sólida (~/~), a função exp ou expoentes apropriados. Italice os símbolos romanos para quantidades e variáveis, mas não os símbolos gregos. Use um traço longo em vez de um hífen para o sinal de menos. Pontue as equações com vírgulas ou pontos quando elas fizerem parte de uma frase, como em:

\begin{equation}
a+b=\gamma\label{eq}
\end{equation}

Certifique-se de que os símbolos em sua equação tenham sido definidos antes ou imediatamente após a equação. Use "Eq.~\eqref{eq}", não "Eq.~\eqref{eq}" ou "equação \eqref{eq}", exceto no início de uma frase: "A equação \eqref{eq} é..."

\subsection{Citações}

Para realizar citações use o 'cite\{nomeDaReferência\}', isso vai permitir que as referências sejam indexadas de forma correta. Observe neste modelo o uso das citações como aqui \cite{al2010image}.

\section{Experimentos}\label{:experiments}

Na seção de experimentos de um artigo, você descreve os detalhes específicos dos experimentos ou estudos realizados como parte da sua pesquisa. Essa seção fornece informações sobre como você coletou os dados e realizou as análises. Aqui estão algumas informações que você pode incluir na seção de experimentos:

\subsection{Objetivos do experimento}
Declare claramente os objetivos do experimento ou estudo em questão. Explique o que você pretende investigar, medir ou comparar por meio do experimento.

Variáveis independentes e dependentes: Identifique as variáveis independentes (aquelas que você manipula ou controla) e as variáveis dependentes (aquelas que você mede ou observa em resposta à manipulação). Descreva as unidades de medida e como as variáveis foram operacionalizadas.

Descreva em detalhes os procedimentos específicos que foram seguidos durante o experimento. Explique como você implementou a manipulação das variáveis independentes, a coleta de dados e qualquer intervenção realizada. Inclua informações sobre o design experimental, a ordem das condições, as tarefas ou estímulos apresentados aos participantes e qualquer etapa de pré-processamento de dados.

\subsection{Instrumentos e materiais}
Descreva os instrumentos, questionários, escalas de medição, equipamentos ou materiais específicos utilizados no experimento. Explique sua validade e confiabilidade, se aplicável, e como eles foram administrados ou apresentados aos participantes.

\subsection{Procedimentos de análise de dados}
Descreva os procedimentos de análise de dados que foram aplicados aos resultados do experimento. Explique as técnicas estatísticas utilizadas para analisar os dados e como elas foram escolhidas para responder às suas perguntas de pesquisa. Mencione quaisquer pressupostos estatísticos e como você tratou dados ausentes ou atípicos.

Certifique-se de que a descrição dos experimentos seja clara e suficientemente detalhada para que outros pesquisadores possam entender e replicar seu estudo. Inclua referências adequadas para instrumentos, escalas ou materiais utilizados, se necessário.

\section{Resultados}\label{:results}

Na seção de resultados de um artigo, você apresenta e descreve os principais achados ou descobertas da sua pesquisa. Essa seção deve ser objetiva e clara, fornecendo uma análise dos dados coletados e respondendo às perguntas de pesquisa. Aqui estão algumas informações que você pode incluir na seção de resultados:

\subsection{Organização dos resultados}
Organize seus resultados de forma lógica e coerente. Geralmente, os resultados são apresentados de acordo com as perguntas de pesquisa ou hipóteses estabelecidas no início do estudo. Você pode optar por usar tabelas, gráficos, figuras ou outros meios visuais para ajudar a apresentar seus resultados de forma clara e concisa.

\subsection{Descrição dos principais achados}
Descreva os principais achados da sua pesquisa em relação às perguntas de pesquisa. Apresente os resultados de forma clara e objetiva, utilizando dados quantitativos ou qualitativos relevantes. Faça referência às estatísticas descritivas, medidas de tendência central, diferenças significativas, correlações ou outros resultados estatísticos relevantes, conforme apropriado.

\subsection{Suporte aos resultados com evidências}
Apresente evidências, como exemplos de dados coletados, citações de entrevistas, transcrições de observações, trechos de documentos analisados, entre outros. Isso ajuda a sustentar seus achados e permite que os leitores compreendam a base dos seus resultados.

\subsection{Comparação com a literatura existente}
Compare seus resultados com estudos anteriores ou com a literatura relevante. Destaque semelhanças e diferenças entre seus achados e o que foi descoberto anteriormente. Discuta se seus resultados estão alinhados ou contradizem as descobertas de outros estudos e explique possíveis explicações para essas discrepâncias.

É importante apresentar os resultados de forma objetiva e precisa, evitando interpretações excessivas ou conclusões não suportadas pelos dados. Utilize uma linguagem clara e concisa, evitando jargões excessivos. Seja transparente e honesto sobre seus resultados, destacando tanto os aspectos positivos quanto os negativos da sua pesquisa.

Dê preferência para apresentar dados tabelados, como na Tabela \ref{tab:resultados}. Todos os elementos não textuais, como tabelas devem ser sempre referenciados no texto.

\begin{table}[htbp]
\caption{Exemplo de tabela para o artigo}
\begin{center}
\begin{tabular}{|c|c|c|c|}
\hline
\textbf{Table}&\multicolumn{3}{|c|}{\textbf{Table Column Head}} \\
\cline{2-4} 
\textbf{Head} & \textbf{\textit{Table column subhead}}& \textbf{\textit{Subhead}}& \textbf{\textit{Subhead}} \\
\hline
copy& More table copy$^{\mathrm{a}}$& &  \\
\hline
\end{tabular}
\label{tab:resultados}
\end{center}
\end{table}

\section{Discussão}\label{:discussion}

Na seção de discussão de um artigo, você interpreta os resultados do seu estudo, relaciona-os ao contexto mais amplo da pesquisa e explora suas implicações teóricas e práticas. A discussão é uma parte crucial do artigo, pois permite que você analise criticamente seus achados e forneça insights adicionais sobre o tema em questão. Aqui estão algumas informações que você pode incluir na seção de discussão:

\subsection{Reafirmação dos objetivos}
Reafirme os objetivos da sua pesquisa e explique como seus resultados contribuem para responder às perguntas de pesquisa estabelecidas no início do artigo. Relacione seus achados aos objetivos do estudo.

\subsection{Comparação com a literatura existente} 
Reforce a comparação entre seus resultados e estudos anteriores ou literatura relevante. Discuta como seus achados se alinham ou se diferenciam do que foi descoberto anteriormente. Explique possíveis explicações para as semelhanças ou diferenças encontradas.

\subsection{Análise dos resultados}
Analise os resultados do seu estudo em profundidade. Identifique padrões, tendências ou relações importantes nos dados. Explique a importância dessas descobertas e ofereça interpretações alternativas ou possíveis explicações para os achados.

\subsection{Implicações teóricas}
Discuta as implicações teóricas dos seus resultados. Explique como seus achados contribuem para o avanço do conhecimento na área e como eles podem apoiar, refinar ou desafiar teorias existentes. Sugira áreas de pesquisa futura que podem surgir a partir dos seus resultados.

É importante ser objetivo, lógico e baseado em evidências ao discutir os resultados. Evite especulações não fundamentadas e seja transparente sobre as limitações do seu estudo. A discussão deve fornecer uma análise aprofundada e perspicaz dos resultados, destacando sua relevância e contribuição para o conhecimento existente na área de estudo.

\section{Conclusão}\label{:conclusion}

Na seção de conclusões de um artigo, você resume os principais pontos discutidos ao longo do estudo e destaca as contribuições e implicações mais importantes da pesquisa. É nessa seção que você oferece uma resposta clara às perguntas de pesquisa e encerra o artigo de maneira concisa. Aqui estão algumas informações que você pode incluir nas conclusões:

\subsection{Recapitulação dos principais resultados}
Resuma brevemente os principais resultados e descobertas do seu estudo. Destaque os achados mais relevantes e significativos que surgiram da sua pesquisa.

\subsection{Resposta às perguntas de pesquisa}
Reafirme como seus resultados respondem às perguntas de pesquisa estabelecidas no início do artigo. Demonstre como sua pesquisa preencheu uma lacuna de conhecimento ou contribuiu para o avanço do campo.

\subsection{Contribuições para o campo}
Destaque as contribuições originais do seu estudo para a área de estudo em questão. Explique como seus resultados ampliam, refinam ou desafiam o conhecimento existente e como eles podem influenciar pesquisas futuras.

\subsection{Limitações do estudo}
Reconheça e discuta as limitações do seu estudo. Seja transparente sobre quaisquer restrições ou desafios que possam ter afetado seus resultados e interpretações. Isso demonstra uma abordagem crítica e honesta em relação à pesquisa.

\subsection{Sugestões para pesquisas futuras}
Ofereça sugestões para pesquisas futuras que possam construir sobre seu trabalho e abordar as limitações identificadas. Identifique áreas de investigação que possam se beneficiar da continuação do seu estudo.

\subsection{Encerramento} Conclua o artigo de forma clara e concisa, destacando a relevância geral do seu trabalho e sua contribuição para o campo de estudo. Evite adicionar novas informações nesta seção e mantenha o foco nas principais conclusões e mensagens do seu estudo.

Lembre-se de que as conclusões devem ser baseadas em evidências sólidas e nos resultados apresentados ao longo do artigo. Elas devem fornecer um fechamento convincente para a pesquisa e enfatizar a importância e o impacto dos seus achados.

\section*{Agradecimentos}

Adicione agradecimentos ao instituições, empresas de fomento.

\bibliographystyle{IEEEtran}
\bibliography{references}

\end{document}
