\section{Trabalhos Relacionados}\label{:relatedwork}

A seção Trabalhos Relacionados ("Related Work") em um artigo tem como objetivo fornecer uma revisão mais detalhada dos estudos anteriores e das pesquisas relacionadas ao tema do seu estudo. Essa seção demonstra que você está familiarizado com a literatura existente e ajuda a posicionar seu trabalho em relação a outros estudos.

\subsection{Visão geral da literatura} 
Inicie com uma visão geral da literatura relevante. Identifique os principais tópicos, abordagens ou teorias discutidas na literatura e explique como essas pesquisas são relevantes para o seu estudo.

\subsection{Trabalhos relacionados}
Discuta estudos anteriores que são diretamente relevantes para o seu trabalho. Descreva brevemente esses estudos, destacando seus objetivos, métodos e principais descobertas. Compare e contraste esses estudos com o seu, ressaltando como o seu trabalho se diferencia ou complementa o que já foi feito. Todos os artigos devem estar referenciados como aqui \cite{b2}. As citações são pontos importantes do trabalho, por que demonstra que o aluno se aprofundou no tema estudado \cite{b4}.

\subsection{Lacunas e limitações}
Identifique as lacunas na literatura existente ou as limitações dos estudos anteriores. Mostre como seu trabalho pretende preencher essas lacunas ou superar essas limitações. Isso ajuda a justificar a importância do seu estudo e a destacar sua contribuição para o campo.

\paragraph{Dica importante}: Dê destaque a estudos recentes que foram publicados na área do seu tema. Demonstre como esses estudos estão alinhados com o seu trabalho e como seu estudo se baseia ou expande essas pesquisas mais recentes.

É importante ser completo ao revisar a literatura relacionada, citando adequadamente os estudos relevantes e fornecendo referências bibliográficas. Isso demonstra que você está construindo seu trabalho sobre uma base sólida de conhecimento existente e contribuindo para a conversa acadêmica.