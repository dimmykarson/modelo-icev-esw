\section{Conclusão}\label{:conclusion}

Na seção de conclusões de um artigo, você resume os principais pontos discutidos ao longo do estudo e destaca as contribuições e implicações mais importantes da pesquisa. É nessa seção que você oferece uma resposta clara às perguntas de pesquisa e encerra o artigo de maneira concisa. Aqui estão algumas informações que você pode incluir nas conclusões:

\subsection{Recapitulação dos principais resultados}
Resuma brevemente os principais resultados e descobertas do seu estudo. Destaque os achados mais relevantes e significativos que surgiram da sua pesquisa.

\subsection{Resposta às perguntas de pesquisa}
Reafirme como seus resultados respondem às perguntas de pesquisa estabelecidas no início do artigo. Demonstre como sua pesquisa preencheu uma lacuna de conhecimento ou contribuiu para o avanço do campo.

\subsection{Contribuições para o campo}
Destaque as contribuições originais do seu estudo para a área de estudo em questão. Explique como seus resultados ampliam, refinam ou desafiam o conhecimento existente e como eles podem influenciar pesquisas futuras.

\subsection{Limitações do estudo}
Reconheça e discuta as limitações do seu estudo. Seja transparente sobre quaisquer restrições ou desafios que possam ter afetado seus resultados e interpretações. Isso demonstra uma abordagem crítica e honesta em relação à pesquisa.

\subsection{Sugestões para pesquisas futuras}
Ofereça sugestões para pesquisas futuras que possam construir sobre seu trabalho e abordar as limitações identificadas. Identifique áreas de investigação que possam se beneficiar da continuação do seu estudo.

\subsection{Encerramento} Conclua o artigo de forma clara e concisa, destacando a relevância geral do seu trabalho e sua contribuição para o campo de estudo. Evite adicionar novas informações nesta seção e mantenha o foco nas principais conclusões e mensagens do seu estudo.

Lembre-se de que as conclusões devem ser baseadas em evidências sólidas e nos resultados apresentados ao longo do artigo. Elas devem fornecer um fechamento convincente para a pesquisa e enfatizar a importância e o impacto dos seus achados.