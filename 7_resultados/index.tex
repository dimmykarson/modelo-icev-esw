\section{Resultados}\label{:results}

Na seção de resultados de um artigo, você apresenta e descreve os principais achados ou descobertas da sua pesquisa. Essa seção deve ser objetiva e clara, fornecendo uma análise dos dados coletados e respondendo às perguntas de pesquisa. Aqui estão algumas informações que você pode incluir na seção de resultados:

\subsection{Organização dos resultados}
Organize seus resultados de forma lógica e coerente. Geralmente, os resultados são apresentados de acordo com as perguntas de pesquisa ou hipóteses estabelecidas no início do estudo. Você pode optar por usar tabelas, gráficos, figuras ou outros meios visuais para ajudar a apresentar seus resultados de forma clara e concisa.

\subsection{Descrição dos principais achados}
Descreva os principais achados da sua pesquisa em relação às perguntas de pesquisa. Apresente os resultados de forma clara e objetiva, utilizando dados quantitativos ou qualitativos relevantes. Faça referência às estatísticas descritivas, medidas de tendência central, diferenças significativas, correlações ou outros resultados estatísticos relevantes, conforme apropriado.

\subsection{Suporte aos resultados com evidências}
Apresente evidências, como exemplos de dados coletados, citações de entrevistas, transcrições de observações, trechos de documentos analisados, entre outros. Isso ajuda a sustentar seus achados e permite que os leitores compreendam a base dos seus resultados.

\subsection{Comparação com a literatura existente}
Compare seus resultados com estudos anteriores ou com a literatura relevante. Destaque semelhanças e diferenças entre seus achados e o que foi descoberto anteriormente. Discuta se seus resultados estão alinhados ou contradizem as descobertas de outros estudos e explique possíveis explicações para essas discrepâncias.

É importante apresentar os resultados de forma objetiva e precisa, evitando interpretações excessivas ou conclusões não suportadas pelos dados. Utilize uma linguagem clara e concisa, evitando jargões excessivos. Seja transparente e honesto sobre seus resultados, destacando tanto os aspectos positivos quanto os negativos da sua pesquisa.

Dê preferência para apresentar dados tabelados, como na Tabela \ref{tab:resultados}. Todos os elementos não textuais, como tabelas devem ser sempre referenciados no texto.

\begin{table}[htbp]
\caption{Exemplo de tabela para o artigo}
\begin{center}
\begin{tabular}{|c|c|c|c|}
\hline
\textbf{Table}&\multicolumn{3}{|c|}{\textbf{Table Column Head}} \\
\cline{2-4} 
\textbf{Head} & \textbf{\textit{Table column subhead}}& \textbf{\textit{Subhead}}& \textbf{\textit{Subhead}} \\
\hline
copy& More table copy$^{\mathrm{a}}$& &  \\
\hline
\end{tabular}
\label{tab:resultados}
\end{center}
\end{table}