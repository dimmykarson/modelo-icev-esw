\section{Discussão}\label{:discussion}

Na seção de discussão de um artigo, você interpreta os resultados do seu estudo, relaciona-os ao contexto mais amplo da pesquisa e explora suas implicações teóricas e práticas. A discussão é uma parte crucial do artigo, pois permite que você analise criticamente seus achados e forneça insights adicionais sobre o tema em questão. Aqui estão algumas informações que você pode incluir na seção de discussão:

\subsection{Reafirmação dos objetivos}
Reafirme os objetivos da sua pesquisa e explique como seus resultados contribuem para responder às perguntas de pesquisa estabelecidas no início do artigo. Relacione seus achados aos objetivos do estudo.

\subsection{Comparação com a literatura existente} 
Reforce a comparação entre seus resultados e estudos anteriores ou literatura relevante. Discuta como seus achados se alinham ou se diferenciam do que foi descoberto anteriormente. Explique possíveis explicações para as semelhanças ou diferenças encontradas.

\subsection{Análise dos resultados}
Analise os resultados do seu estudo em profundidade. Identifique padrões, tendências ou relações importantes nos dados. Explique a importância dessas descobertas e ofereça interpretações alternativas ou possíveis explicações para os achados.

\subsection{Implicações teóricas}
Discuta as implicações teóricas dos seus resultados. Explique como seus achados contribuem para o avanço do conhecimento na área e como eles podem apoiar, refinar ou desafiar teorias existentes. Sugira áreas de pesquisa futura que podem surgir a partir dos seus resultados.

É importante ser objetivo, lógico e baseado em evidências ao discutir os resultados. Evite especulações não fundamentadas e seja transparente sobre as limitações do seu estudo. A discussão deve fornecer uma análise aprofundada e perspicaz dos resultados, destacando sua relevância e contribuição para o conhecimento existente na área de estudo.