\section{Abordagem}\label{:approach}

Na seção de abordagem de um artigo, você descreve a perspectiva teórica, a estrutura conceitual ou a abordagem metodológica que você adotou para abordar seu objeto de estudo. Essa seção permite que os leitores compreendam o enfoque teórico ou prático subjacente ao seu trabalho. A seguir estão algumas informações que você pode incluir na seção de abordagem.

\subsection{Perspectiva teórica}
Descreva a perspectiva teórica que fundamenta seu estudo. Explique o arcabouço conceitual ou a teoria que serve de base para a pesquisa. Destaque os principais conceitos, modelos ou teorias que você está utilizando para analisar ou interpretar seus dados.

Se o seu estudo envolve a construção de uma estrutura conceitual ou modelo, explique-o nesta seção. Descreva os elementos-chave da estrutura e explique como eles se relacionam entre si. Mostre como essa estrutura conceitual ajudará a responder às perguntas de pesquisa ou a abordar o problema proposto.

\subsection{Metodologia adotada} Se o seu estudo é orientado para a prática ou possui uma abordagem metodológica específica, explique-a nesta seção. Por exemplo, se você está usando uma abordagem de estudo de caso, pesquisa-ação, pesquisa participativa ou outra metodologia específica, descreva como essa abordagem orientou sua coleta de dados e análise.

Explique por que você escolheu essa abordagem em particular e como ela se alinha com seus objetivos de pesquisa. Discuta as vantagens e as limitações dessa abordagem específica em relação ao seu estudo e como ela pode contribuir para responder às suas perguntas de pesquisa.

É importante fornecer detalhes suficientes para que os leitores possam compreender claramente a abordagem teórica ou metodológica que você adotou. Use exemplos ou ilustrações, quando apropriado, para ajudar a esclarecer a abordagem e sua relevância para o estudo.

\subsection{Figuras e Tabelas}
\paragraph{Posicionamento de Figuras e Tabelas} Posicione as figuras e tabelas no topo e na parte inferior das colunas. Evite colocá-las no meio das colunas. Figuras e tabelas grandes podem se estender por ambas as colunas. As legendas das figuras devem estar abaixo das figuras; os cabeçalhos das tabelas devem aparecer acima das tabelas. Insira figuras e tabelas após serem mencionadas no texto. Use a abreviação "Fig.~\ref{fig:figbase}" mesmo no início de uma frase.

É importante ilustrar sua abordagem com imagens sempre que apropriado. Imagens, como gráficos, tabelas, diagramas ou esquemas, podem ser uma forma eficaz de visualizar conceitos complexos, padrões de dados, relações entre variáveis ou processos metodológicos. As imagens devem ser sempre referencias como quando usamos a Figura \ref{fig:figbase}.

\begin{figure}[htbp]
    \centering
    \includegraphics{images/fig1.png}
    \caption{Nunca esqueça de colocar uma legenda para cada imagem associada ao seu trabalho.}
    \label{fig:figbase}
\end{figure}

\subsection{Equações}
Numere as equações consecutivamente. Para tornar suas equações mais compactas, você pode usar a barra sólida (~/~), a função exp ou expoentes apropriados. Italice os símbolos romanos para quantidades e variáveis, mas não os símbolos gregos. Use um traço longo em vez de um hífen para o sinal de menos. Pontue as equações com vírgulas ou pontos quando elas fizerem parte de uma frase, como em:

\begin{equation}
a+b=\gamma\label{eq}
\end{equation}

Certifique-se de que os símbolos em sua equação tenham sido definidos antes ou imediatamente após a equação. Use "Eq.~\eqref{eq}", não "Eq.~\eqref{eq}" ou "equação \eqref{eq}", exceto no início de uma frase: "A equação \eqref{eq} é..."

\subsection{Citações}

Para realizar citações use o 'cite\{nomeDaReferência\}', isso vai permitir que as referências sejam indexadas de forma correta. Observe neste modelo o uso das citações como aqui \cite{al2010image}.