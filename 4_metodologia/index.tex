\section{Metodologia}\label{:methodoloy}

A seção de metodologia em um artigo descreve o processo que você seguiu para conduzir o seu estudo ou pesquisa. Ela deve ser detalhada o suficiente para permitir que outros pesquisadores reproduzam o seu trabalho e avaliem a validade dos resultados. Aqui estão algumas informações que você pode incluir na seção de metodologia:

\subsection{Desenho do estudo}
Descreva o desenho do estudo, ou seja, a abordagem geral adotada para coletar dados e responder às perguntas de pesquisa. Por exemplo, você pode mencionar se o estudo é qualitativo, quantitativo, experimental, observacional, etc.

Participantes ou amostra: Descreva as características dos participantes ou da amostra envolvida no estudo. Inclua informações como o tamanho da amostra, critérios de inclusão e exclusão, método de seleção dos participantes, e quaisquer outras informações relevantes sobre as pessoas envolvidas no estudo.

Coleta de dados: Explique como os dados foram coletados. Isso pode incluir detalhes sobre os instrumentos de coleta de dados utilizados, como questionários, entrevistas, observações ou medidas objetivas. Descreva também o local e o período de coleta de dados.

\subsection{Procedimentos}
Descreva os procedimentos específicos seguidos durante o estudo. Isso pode incluir instruções dadas aos participantes, detalhes sobre as condições experimentais, sequência de atividades, cronograma e qualquer intervenção realizada.

Variáveis e medidas: Liste e descreva as variáveis do estudo, especificando como elas foram medidas ou operacionalizadas. Explique as escalas de medição utilizadas, bem como a validade e confiabilidade dos instrumentos de medição, quando aplicável.

\subsection{Análise de dados}

Descreva as técnicas estatísticas ou métodos de análise de dados que foram aplicados aos dados coletados. Explique como as variáveis foram analisadas e quaisquer suposições subjacentes aos métodos estatísticos utilizados.


\subsection{Informações sobre o artigo}

\subsubsection{Autores e Afiliações}
\textbf{O arquivo de classe é projetado para, mas não limitado a, seis autores.} É necessário no mínimo um autor. Os nomes dos autores devem ser listados da esquerda para a direita e, em seguida, descendo para a próxima linha. Essa sequência de autores será usada em futuras citações e pelos serviços de indexação. Os nomes não devem ser listados em colunas ou agrupados por afiliação.