\section{Introdução}

A introdução de um artigo é uma seção importante que estabelece o contexto, a relevância e os objetivos do estudo. Ela deve fornecer ao leitor uma visão geral do tema, despertando seu interesse e estabelecendo as bases para o restante do artigo. Aqui estão alguns elementos que geralmente são incluídos em uma introdução:

\subsection{Contextualização}
Apresente o tema geral do artigo, fornecendo informações de fundo e destacando a importância do assunto. Explique por que o tema é relevante e qual a relevância acadêmica ou prática do estudo.

\subsection{Revisão da literatura}
Faça uma breve revisão dos trabalhos anteriores e das pesquisas relacionadas ao tema do seu estudo. Isso ajudará a situar seu trabalho dentro do contexto existente, mostrando como sua pesquisa se diferencia ou se baseia em estudos anteriores.

\subsection{Lacuna na literatura}
Identifique uma lacuna na pesquisa existente que seu estudo pretende preencher. Descreva como seu trabalho se encaixa nessa lacuna e como ele pode contribuir para o conhecimento atual sobre o assunto.

\subsection{Objetivos da pesquisa}
Declare claramente os objetivos e as perguntas de pesquisa do seu estudo. Explique o que você pretende alcançar e quais questões você está tentando responder. Essa parte deve ser específica e concisa.

\subsection{Estrutura do artigo}
Forneça uma breve visão geral da estrutura do artigo, descrevendo como as seções subsequentes estão organizadas e o que será abordado em cada uma delas. Isso ajudará o leitor a entender a organização do seu trabalho. Na Seção \ref{:relatedwork} possui uma revisão mais detalhada dos estudos anteriores e das pesquisas relacionadas ao tema do seu estudo. A Seção \ref{:methodoloy} descreve o processo que foi seguido para conduzir o estudo ou pesquisa. A Seção \ref{:approach} descreve a perspectiva teórica, a estrutura conceitual ou a abordagem metodológica que você adotou para abordar seu objeto de estudo. Em seguida a seção \ref{:experiments} descreve-se os detalhes específicos dos experimentos ou estudos realizados como parte da pesquisa. Na Seção \ref{:results} apresenta-se e descreve-se os principais achados ou descobertas da pesquisa. Uma interpretação dos resultados do estudo, relaciona-os ao contexto mais amplo da pesquisa e suas implicações teóricas e práticas são discutidos na Seção \ref{:discussion}. Por fim, na Seção \ref{:conclusion}  resume-se os principais pontos discutidos ao longo do estudo e destaca as contribuições e implicações mais importantes da pesquisa.

É importante lembrar que a introdução deve ser clara, sucinta e direta ao ponto. Evite informações desnecessárias ou excessivamente detalhadas. Seja objetivo ao estabelecer o contexto e garantir que a introdução seja interessante o suficiente para cativar a atenção do leitor.